\documentclass[a4paper,12pt]{article}
%%%%%%%%%%%%%%%%%%%%%%%%%%%%%%%%%%%%%%%%%%%%%%%%%%%%%%%%%%%%%%%%%%%%%%%%%%%%%%%%%%%%%%%%%%%%%%%%%%%%%%%%%%%%%%%%%%%%%%%%%%%%%%%%%%%%%%%%%%%%%%%%%%%%%%%%%%%%%%%%%%%%%%%%%%%%%%%%%%%%%%%%%%%%%%%%%%%%%%%%%%%%%%%%%%%%%%%%%%%%%%%%%%%%%%%%%%%%%%%%%%%%%%%%%%%%
\usepackage{eurosym}
\usepackage{vmargin}
\usepackage{amsmath}
\usepackage{graphics}
\usepackage{epsfig}
\usepackage{subfigure}
\usepackage{fancyhdr}

\setcounter{MaxMatrixCols}{10}
%TCIDATA{OutputFilter=LATEX.DLL}
%TCIDATA{Version=5.00.0.2570}
%TCIDATA{<META NAME="SaveForMode"CONTENT="1">}
%TCIDATA{LastRevised=Wednesday, February 23, 201113:24:34}
%TCIDATA{<META NAME="GraphicsSave" CONTENT="32">}
%TCIDATA{Language=American English}

\pagestyle{fancy}
\setmarginsrb{20mm}{0mm}{20mm}{25mm}{12mm}{11mm}{0mm}{11mm}
\lhead{MA4128} \rhead{Kevin O'Brien} \chead{Week 6} %\input{tcilatex}

\begin{document}

\tableofcontents
\newpage
%------------------------------------------------------------------%
\newpage
\section{Two-Step Cluster}
When you have a really large data set or you need a clustering procedure that can rapidly form clusters on the basis of either categorical or continuous data, neither of the previous two procedures are entirely appropriate. Hierarchical clustering requires a matrix of distances between all pairs of cases, and k-means requires shuffling cases in and out of clusters and knowing the number of clusters in advance.

The Two-Step Cluster Analysis procedure was designed for such applications. The name two-step clustering is already an indication that the algorithm is based on a two-stage approach: In the first stage, the algorithm undertakes a procedure that is very similar to the k-means algorithm. Based on these results, the two-step
procedure conducts a modified hierarchical agglomerative clustering procedure that
combines the objects sequentially to form homogenous clusters. This is done by
building a so-called \textbf{\textit{cluster feature tree}} whose \textbf{\textit{leaves}} represent distinct objects in the dataset. The procedure can handle categorical and continuous variables simultaneously
and offers the user the flexibility to specify the cluster numbers as well as
the maximum number of clusters, or to allow the technique to automatically choose
the number of clusters on the basis of statistical evaluation criteria.

The Two-Step Cluster Analysis requires only one pass of data
(which is important for very large data files).

Additionally, the procedure indicates each variable’s
importance for the construction of a specific cluster. These desirable features make
the somewhat less popular two-step clustering a viable alternative to the traditional
methods.
\subsection{Measures of Fit}
Two-Step Cluster Analysis guides the decision of how many clusters to retain from the data by
calculating measures-of-fit such as \textbf{\textit{Akaike’s Information Criterion (AIC)}} or \textbf{\textit{Bayes Information Criterion (BIC)}}.

These are relative measures of goodness-of-fit and are used to compare different
solutions with different numbers of segments.(``Relative" means that these criteria
are not scaled on a range of, for example, 0 to 1 but can generally take any value.)
Compared to an alternative solution with a different number of segments, smaller
values in AIC or BIC indicate an increased fit.

SPSS computes solutions for different
segment numbers (up to the maximum number of segments specified before) and
chooses the appropriate solution by looking for the smallest value in the chosen
criterion. However, which criterion should we choose? AIC is well-known for
overestimating the correct number of segments, while BIC has a slight tendency
to underestimate this number.

Thus, it is worthwhile comparing the clustering
outcomes of both criteria and selecting a smaller number of segments than
actually indicated by AIC. Nevertheless, when running two separate analyses,
one based on AIC and the other based on BIC, SPSS usually renders the same
results.
\end{document}

The clustering algorithm is based on a distance measure that gives the best results if all variables are independent, continuous variables have a normal distribution (or categorical variables have a multinomial distribution). This is seldom the case in practice, but the algorithm is thought to behave reasonably well when the assumptions are not met.

Because cluster analysis does not involve hypothesis testing and calculation of observed significance levels, other than for descriptive follow-up, it's perfectly acceptable to cluster data that may not meet the assumptions for best performance.

The final outcome may depend on the order of the cases in the file. To minimize the effect, arrange the cases in random order. Sort them by the last digit of their ID numbers or something similar.

\subsection{Step 1: Pre-clustering: Making Little Clusters}
The first step of the two-step procedure is formation of pre-clusters. The goal of pre-clustering is to reduce the size of the Distance matrix (the matrix that contains distances between all
possible pairs of cases). Pre-clusters are just clusters of the original cases that are used in place of the raw data in the hierarchical clustering. As a case is read, the algorithm
decides, based on a distance measure, if the current case should be merged with a previously formed pre-cluster or start a new precluster.

When preclustering is complete, all cases in the same precluster are treated as a single entity. The size of the distance matrix is no longer dependent on the number of cases but on the number of preclusters.

\subsection{Step 2: Hierarchical Clustering of Preclusters}
In the second step, SPSS uses the standard hierarchical clustering algorithm on the
preclusters. Forming clusters hierarchically lets you explore a range of solutions with
different numbers of clusters.
Tip: The Options dialog box lets you control the number of preclusters. Large numbers
of preclusters give better results because the cases are more similar in a precluster;
however, forming many preclusters slows the algorithm.

Some of the options you can specify when using two-step clustering are:
\textbf{Standardization:} The algorithm will automatically standardize all of the variables
unless you override this option.

\textbf{Distance measures:} If your data are a mixture of continuous and categorical variables,
you can use only the log-likelihood criterion. The distance between two clusters
depends on the decrease in the log-likelihood when they are combined into a single
cluster. If the data are only continuous variables, you can use the Euclidean
distance between two cluster centers. Depending on the distance measure selected,
cases are assigned to the cluster that leads to the largest log-likelihood or to the cluster
that has the smallest Euclidean distance.


\textbf{Number of clusters:} You can specify the number of clusters to be formed, or you can let
the algorithm select the optimal number based on either the Schwarz Bayesian
Criterion or the Akaike information criterion.


\textbf{Outlier handling:} You have the option to create a separate cluster for cases that don't fit
well into any other cluster.


\textbf{Range of solutions:} You can specify the range of cluster solutions that you want to see.

\end{document} 